% This document is a test example of an article document in LaTeX. 

% The class determine the type of document and any other aspects of it

\documentclass[12pt, letterpaper, twoside]{article}

% This package encodes the document

\usepackage[utf8]{inputenc}

% These allow images to be embedded

\usepackage{graphicx}
\graphicspath{ {../Images/} }

% This package gives access to more complex math expressions

\usepackage{amsmath}

% This package puts a full newline between each paragraph

\usepackage{parskip}
\setlength{\parindent}{15pt}

% This establishes the header

\title{First document!}
\author{Vyoma Jani \thanks{Helped by that PDF that I Googled}}
% \date{December 25, 2019}

% This is where the body of the document starts

\begin{document}

\maketitle

% This creates an abstract for the document

\begin{abstract}
This is the abstract, which goes at the top of the document. It introduces the subject, which, in this case, is testing out example tags in \LaTeX{}.
\end{abstract}

First \LaTeX{} document!

So far, this example has used classes, packages, titles, authors, and dates! We have learned how to \textbf{bold}, \textit{italicize}, and \underline{underline}. We can \emph{emphasize} certain things, \textit{or we can even \emph{not emphasize} other things}.

We have also learned how to add images:

\begin{figure}[h]
	\centering\includegraphics[width=0.25\textwidth]{Sun}
	\caption{The sun}
	\label{fig:mesh1}
\end{figure}

Now, if we want to reference Figure \ref{fig:mesh1}, we just use a command!

We also learned about lists!

\begin{itemize}
	\item This is an unordered list
	\item It has bullets
\end{itemize}

\begin{enumerate}
	\item This is an ordered list
	\item It is numbered
\end{enumerate}

And we learned how to add math expressions to the text. 

To write a subscript, use an underline, like in $a_b$. For a superscript, use the exponent symbol, like in $a^b$. 

We can put expressions inline using three different delimiters: \(E=mc^2\), $\int_0^1 \frac{x}{2} dx$, or \begin{math} \omega \end{math}. 

We can also put expressions in display mode using the delimiters: \[ \sin(\delta) \] or \begin{equation}
\Omega\end{equation} which adds a number since it's an equation, or \begin{displaymath}
\cos(\Delta)\end{displaymath}
There is also $$\log(x)$$ but this is not recommended. 

It's interesting \dots you can do so much with \LaTeX{}!

\end{document}
